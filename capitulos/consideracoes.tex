\chapter{Considerações Finais}

Nesta seção iremos pontuar alguns resultados obtidos durante o desenvolvimento desse trabalho. Primeiramente, o autor estabeleceu um bom conhecimento sobre processamento de linguagem natural, incluindo técnicas de pré-processamento e aprendizado de máquina, o que possibilitou a aplicação desses conceitos no desenvolvimento das aplicações propostas nesse trabalho.

Para as contribuições tecnológicas, foi construída uma versão inicial da biblioteca \textit{pygov-br}, que facilita o consumo do \textit{webservice} da Câmara dos Deputados em aplicações \textit{Python}, assim como a persistência dessas informações em banco dados, através da aplicação \textit{Django} que faz parte da \textit{pygov-br}.

Iniciou-se o desenvolvimento da aplicação \textit{web}, com a implementação dos algoritmos de classificação temática e sua principal funcionalidade, que consiste em poder visualizar o perfil temático de um parlamentar. Entretanto, apenas a visualização dos dados organizados por estado foi implementado. As outras funcionalidades serão implementadas pela equipe de desenvolvimento do Laboratório Hacker da Câmara dos Deputados.

\section{Perspectivas Futuras}

A partir do resultado desse trabalho podemos aplicar outras formas de aprendizado de máquina sobre o perfil temático dos deputados. Uma opção é a utilização do \textit{k-means} para determinar a proximidade dos parlamentares de acordo com os temas abordados por eles.

Além disso, várias funcionalidades e melhorias na experiência de usuário da aplicação \textit{web} podem ser implementadas, como buscar diretamente um parlamentar pelo nome, outras formas de organização dos dados (por gênero, partido, tema, etc), melhorar a forma de navegação no sistema, entre outras.

A partir do momento que os dados disponíveis na nova \textit{API} da Câmara dos Deputados contemplarem os dados necessários para o funcionamento da aplicação, pretende-se atualizar a biblioteca \textit{pygov-br} para o novo modelo.
