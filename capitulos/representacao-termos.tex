\subsection{Valores no BOW}
\label{sec:representação_dos_termos}

Apresentamos o modelo do BOW associando um valor \(a_i\) a cada palavra que representa, como a quantidade de vezes que o termo aparece em um texto. Existem, no entanto, outras formas de representação expostas a seguir.

\subsubsection{\textit{Boolean}}
\label{ssub:bag-boolean}

Essa medida associa valores binários para os termos presentes nos documentos, onde o valor de \(a_{i}\) é 0 quando o termo não aparece nenhuma vez ou 1 quando aparece uma ou mais vezes. Essa medida simples é muitas vezes utilizada para a análise de documentos pequenos, como frases e parágrafos.

\subsubsection{\textit{Term Frequency}}
\label{ssub:baf-tf}

A medida \textit{term frequency} considera a quantidade de ocorrências do termo \(t\) dentro do documento, ao contrário da medida mencionada anteriormente \cite{buckley1988}
%
\begin{equation}
f(t)=\frac{n_t}{N},
\end{equation}
%
onde \(n_t\) é a quantidade de vezes que o termo \(t\) aparece dentro do documento e \(N\) a quantidade total de termos do documento.

\subsubsection{\textit{Term Frequency - Inverse Document Frequency (TF-IDF)}}
\label{ssub:baf-tfidf}

Alguns termos comuns podem aparecer na maioria dos documentos sem fornecer informações úteis em uma tarefa de mineração de textos. Para diminuir a influência destes termos, é possível utilizar um fator de ponderação, para que os termos que aparecem na maioria dos documentos tenham valores numéricos menores do que aqueles que raramente aparecem \cite{pretext}. Segundo \citeonline{jones1972}, a especificidade de um termo pode ser quantificada por uma função inversa do número de documentos em que ele ocorre. Uma alternativa comum é utilizar uma função que varia entre \(0\) e \(log N_d\), onde \(N_d\) é o número total de documentos e \(d(t)\) a quantidade de documentos nos quais o termo \(t\) aparece ao menos uma vez:
%
\begin{equation}
i(t)=log \frac{N_d}{d(t)}
\end{equation}
%
Portanto, o valor final de \(a_{i}\) é dado pela equação:
%
\begin{equation}
f(t)=\frac{n_t}{N} \cdot log \frac{N_d}{d(t)},
\end{equation}
%
onde \(\frac{n_t}{N}\) é a frequência do termo dentro do texto e \(log \frac{N_d}{d(t)}\) sua taxa de ponderação.

Existem outras formas de ponderação que penalizam os termos mais comuns de formas mais ou menos extremas.
O formato logaritmo possui a característica de anular termos que apareçam em todos os documentos, considerando-os
portanto, como totalmente não-informativos.
