\section{Proposta de Desenvolvimento}

\subsection{Ferramentas e Tecnologias}

Para auxiliar no desenvolvimento, foram escolhidas as seguintes tecnologias:

\begin{itemize}
    \item \textbf{Linguagem de Programação:} devido ao tamanho da comunidade, grande utilização na área de aprendizado de máquina, possibilidade de desenvolvimento \textit{web} e conhecimento prévio, a linguagem a ser utilizada para o desenvolvimento das aplicações do presente trabalho será \textit{Python}.
    \item \textbf{\textit{Frameworks}:} para o desenvolvimento da aplicação \textit{web}, será utilizado o \textit{Django}. Para trabalhar com o processamento de linguagem natural e aprendizado de máquina aplicado, serão utilizadas as bibliotecas \textit{plagiarism} e \textit{texblob}, que encapsula a biblioteca \textit{NLTK}.
    \item \textbf{Gerenciador de Repositórios de Código:} para gerenciar as diversas versões dos códigos das aplicações desenvolvidas nesse trabalho, a ferramenta utilizada será o sistema de controle de versões \textit{Git} e o serviço de \textit{web hosting} compartilhado \textit{GitHub}.
    \item \textbf{Documentação do Projeto:} para realizar a documentção do projeto, será utilizada a biblioteca \textit{python sphinx}, que possibilita a extração de documentação por meio de \textit{docstrings} no código, e o próprio \textit{GitHub}, através de \textit{readmes} no repositório.
    \item \textbf{Gerenciamento de Tarefas:} para o controle de tarefas que serão executadas, estão em execução ou fazem parte do projeto, será utilizado o sistema de \textit{issues} e quadro de projetos do \textit{GitHub}.
\end{itemize}

\subsection{\textit{Pygov-br}}

\textit{Pygov-br} é uma biblioteca \textit{python} cujo objetivo é centralizar o consumo de \textit{APIs} e \textit{webservices} governamentais brasileiros. Além dos dados, a biblioteca também irá fornecer um conjunto de \textit{plugins} para os principais \textit{frameworks} para desenvovimento \textit{web}, facilitando ainda mais a utilização dos dados abertos do governo brasileiro, já que para utilizá-los será necessário apenas algumas linhas de código.

