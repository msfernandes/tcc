\section{Obtenção dos Dados}
\label{obtencao-dados}

Todos os dados utilizados para análise foram obtidos através do portal de dados abertos da Câmara dos Deputados\footnote{http://www.camara.leg.br/transparencia/dados-abertos}, que é dividido em duas partes: dados legislativos e dados referentes à cota parlamentar, que não foram utilizados nesse trabalho. Os dados legislativos estão relacionados à informações sobre deputados, órgãos legislativos, proposições, sessões plenárias e reuniões de comissões.

\subsection{\textit{Webservice} da Câmara dos Deputados}

Atualmente, o \textit{Webservice} da Câmara dos Deputados é estruturado de acordo com os padrões \textit{SOAP} (\textit{Simple Object Access Protocol}, em português Protocolo Simples de Acesso a Objetos), que se baseiam na linguagem de marcação \textit{XML} e utilizam, principalmente, chamada de procedimento remoto (\textit{RPC}) e protocolo de transferência de hipertexto (\textit{HTTP}) para a transmissão das mensagens \cite{soap2007}.


No entanto, o \textit{webservice} possui alguns aspectos que podem ser melhorados. Como os dados são fornecidos utilizando o formato \textit{XML}, eles não são ``tipados'', ou seja, independente do tipo (inteiro, data, texto, etc) eles são representados com \textit{strings}. Alguns dados são ambíguos, como os referentes aos deputados, onde existem ``ideCadastro'' e ``idParlamentar'', que são utilizados como parâmetros de entrada de requisições distintas. Outro problema é que requisições comuns precisam ser feitas indiretamente pois não agrega conteúdos com \textit{queries} relacionais, como normalmente são as \textit{API REST}. Além disso, o inteiro teor dos discursos parlamentares estão disponíveis apenas em formato \textit{RTF}, o que dificulta um pouco a utilização dos mesmos.

No momento de escrita desse trabalho, se encontra em desenvolvimento uma nova versão do portal de dados abertos da Câmara dos Deputados, que está em fase de testes e já pode ser acessada pela sociedade\footnote{https://dadosabertos.camara.leg.br/}, porém ainda não possui os mesmos dados disponíveis na versão anterior. O novo \textit{webservice} segue os padrões REST e possibilita a escolha do formato de retorno, podendo ser em \textit{XML} ou em  \textit{JSON}. Além disso, visa corrigir os problemas encontrados na versão anterior (alguns mencionados no parágrafo acima), bem como aumentar a quantidade de dados disponíveis. Uma das promessas é disponibilizar o texto completo das proposições, que hoje só é disponível via \textit{PDF} sendo que alguns são apenas imagens \textit{scaneadas} dos documentos físicos.

A estrutura do \textit{webservice} da Câmara dos Deputados pode ser encontrada no apêndice \ref{estrutura-webservice}
