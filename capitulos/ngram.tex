\subsection{Modelo \textit{N-Gram}}

O modelo de representação de textos através de um conjunto de palavras pode limitar qualitativamente a análise, já que termos compostos não são considerados. Por exemplo, ``Santa Catarina'', um estado brasileiro, possui um significado completamente diferente quando as palavaras ``Santa'' (mulher canonizada), e ``Catarina'' (nome feminino) são analisadas separadamente.

Um \(n\)-grama é uma sequência de \(n\) elementos dentro de um texto. Os elementos podem ser palavras, sílabas, letras ou qualquer outra base. É comum usar as denominações unigrama, bigrama e trigrama para \(n\)-gramas de 1, 2 ou 3 elementos. Usando a frase ``Eu não gostei desse filme'' como exemplo, temos os seguintes unigramas: ``eu'', ``não'', ``gostei'', ``desse'' e ``filme''. Os seguintes bigramas: ``eu não'', ``não gostei'', ``gostei desse'' e ``desse filme''. E os seguintes trigramas: ``eu não gostei'', ``não gostei desse'' e ``gostei desse filme''.
