\subsection{Modelo \textit{N-Gram}}

O modelo de representação de textos através de um conjunto de palavras pode limitar qualitativamente a análise que será realizada, já que frases não são consideradas. Por exemplo, ``Santa Catarina'', um estado brasileiro, possui um significado completamente diferente quando as palavaras ``Santa'' (mulher canonizada), e ``Catarina'' (nome feminino) são analisadas separadamente.

Um \(n\)-grama é uma sequência de \(n\) elementos dentro de um texto. Os elementos podem ser palavras, sílabas, letras ou qualquer outra base. Um \(n\)-grama de tamanho 1 é chamado de unigrama, de tamanho 2, bigrama, e de tamanho 3, trigrama. Sequências com 4 ou mais elementos são chamados de \(n\)-gramas. Usando a frase ``Eu não gostei desse filme'' como exemplo, temos os seguintes unigramas: ``eu'', ``não'', ``gostei'', ``desse'' e ``filme''. Os seguintes bigramas: ``eu não'', ``não gostei'', ``gostei desse'' e ``desse filme''. E os seguintes trigramas: ``eu não gostei'', ``não gostei desse'' e ``gostei desse filme''.
