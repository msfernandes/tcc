\section{Ferramentas e Tecnologias}
\label{ferramentas}

Para auxiliar no desenvolvimento, foram escolhidas as seguintes tecnologias:

\subsection{Linguagem de Programação}

Devido ao tamanho da comunidade, grande utilização na área de aprendizado de máquina, possibilidade de desenvolvimento \textit{web} e conhecimento prévio do autor e orientador, decidiu-se utilizar a linguagem \textit{Python}\footnote{https://www.python.org} para o desenvolvimento das aplicações do presente trabalho.


\subsection{\textit{Frameworks} e Bibliotecas}

O desenvolvimento da aplicação \textit{web}, utilizará o \textit{framework} \textit{Django}\footnote{https://www.djangoproject.com}, o que implica na uso da arquitetura \textit{MVT} (\textit{Model View Template}). Similar ao \textit{MVC}, no \textit{MVT} o ciclo começa por uma ação do usuário, a View notifica a Model, para que seu estado seja atualizado, a Model efetua as modificações necessárias e alerta as suas dependências que foi alterada, assim a Template consulta o novo estado da Model, e atualiza a sua visualização.

Para trabalhar com o processamento de linguagem natural e aprendizado de máquina aplicado, serão utilizadas as bibliotecas \textit{plagiarism}\footnote{https://github.com/fabiommendes/plagiarism} e \textit{texblob}\footnote{https://textblob.readthedocs.io}, que encapsula a biblioteca \textit{NLTK}\footnote{http://www.nltk.org}. Alguns cálculos são implementados utilizando o \textit{stack} científico do \textit{Python}, que inclui o \textit{numpy}\footnote{http://www.numpy.org}, \textit{scipy}\footnote{https://www.scipy.org}, \textit{matplotlib}\footnote{http://matplotlib.org} e \textit{sklearn}\footnote{http://scikit-learn.org}.


\subsection{Gerenciador de Repositórios de Código}

O gerenciamento de versões dos códigos das aplicações desenvolvidas utiliza o \textit{Git}\footnote{https://git-scm.com} e o serviço de \textit{web hosting} compartilhado \textit{GitHub}\footnote{https://github.com}. Além disso, os pacotes \textit{Python} são enviados para o \textit{PyPI}\footnote{https://pypi.python.org/pypi}, sendo facilmente instaláveis através do comando \textit{pip}\footnote{https://pip.pypa.io}.


\subsection{Documentação do Projeto}

A documentação do projeto, realizada utilizando a biblioteca \textit{Python sphinx}\footnote{http://www.sphinx-doc.org}, que possibilita a extração de documentação por meio de \textit{docstrings} no código e arquivos \textit{.rst}, e o próprio \textit{GitHub}, através de \textit{readmes} no repositório.

\subsection{Gerenciamento de Tarefas}

O controle de tarefas executadas ou em execução é feito utilizando o sistema de \textit{issues} e quadro de projetos do \textit{GitHub}.


