\chapter{Introdução}

\section{Contextualização}
\label{sec:contextualização}

O desenvolvimento de novas ferramentas de interação entre governo e sociedade é fundamental para o avanço da democracia e a disponibilização de dados governamentais possibilita que novas aplicações surjam, trazendo novas formas de utilização e interpretação destes dados \cite{consegi2011}. O presente trabalho utiliza dados disponíveis publicamente pela Câmara dos Deputados Federal para analisar textos de discursos e proposições parlamentares utilizando técnicas de processamento de linguagem natural e aprendizagem de máquina. O objetivo da análise é determinar eixos temáticos para cada deputado e avaliar o alinhamento entre os temas dos discursos e das propostas encaminhadas.

No contexto governamental, os Dados Abertos\footnote{O conceito para Dado Aberto considerado neste trabalho é o definido pela \citeonline{open_knowledge}, que estabele que um dado (ou um conhecimento) é aberto quando estiver livre para uso, reuso e redistribuição. Ou seja, a informação deve estar disponível a todos, sem restrições de \textit{copyright}, patentes ou outros mecanismos de controle. Além disso, esses dados devem ser independentes de tecnologia, baseados em formatos padronizados e desvinculados das ferramentas que os originaram. Os dados devem permitir sua manipulação por máquinas e possuir metadados que permitam identificar sua natureza, origem e qualidade \cite{diniz2010}.} fortalecem três características indispensáveis para a democracia: transparência, participação e colaboração \cite{consegi2011}. Transparência tem o papel de informar a sociedade sobre as ações que estão sendo tomadas ou que serão tomadas pelo governo. Participação permite que os cidadãos auxiliem o poder público a elaborar políticas mais eficazes. Finalmente, a colaboração entre a sociedade, diferentes níveis de governo e a iniciativa privada permitem aprimorar a eficácia do Estado.

De acordo com a Lei n\degree 12.527/2011, também conhecida como Lei de Acesso à Informação, qualquer cidadão, sem necessidade de justificativa, pode solicitar dados ou informações à qualquer órgão ou entidade pública dos poderes Executivo, Legislativo e Judiciário, além do Ministério Público, nas esferas Federal, Estadual e Municipal \cite{lei_acesso_informacao}. Para atender à lei mencionada anteriormente, a \citeonline{camara_dados_abertos} criou um portal que tem como objetivo disponibilizar dados brutos para a utilização em aplicações desenvolvidas pelos cidadãos e entidades da sociedade civil que permitam a percepção mais efetiva das atividades parlamentares, .

A publicação de dados pressupõe o uso de tecnologias que garantam que eles possam ser acessados e reutilizados por máquinas. Apesar de não garantir que os dados estarão disponíveis em um formato conveniente de uso imediato, possibilita o cruzamento de diferentes bases dados e a exibição destes de forma que possam ser melhor apresentados à sociedade \cite{diniz2010}.

\section{Objetivo}
\label{sec:objetivo}

O objetivo desse trabalho é utilizar técnicas de processamento de linguagem natural para, através da análise dos discursos e proposições dos parlamentares, determinar um perfil temático para os deputados, bem como evidenciar o termos mais utilizados em seus discursos e proposições e, assim, permitir que o cidadão veja, de forma comparativa, o que seus representantes no parlamento mais dizem em seus discursos e o que mais dizem em suas proposições.

\subsection{Contribuições Tecnológicas}
\label{sub:contribuicoes_tecnologicas}

\begin{itemize}
    \item Implementar biblioteca Python para consumo de dados abertos governamentais, com foco nos dados abertos da Câmara dos deputados.
    \item Implementar aplicação Django para utilização e persistência dos dados abertos governamentais, também com foco nos dados aberto da Câmara dos deputados.
    \item Implementar sistema web de comparação temática entre discursos e proposições parlamentares, a ser detalhado no decorrer deste trabalho.
\end{itemize}

\subsection{Contribuições Científicas}
\label{sub:contribuições_científicas}

\begin{itemize}
    \item Estudo teórico sobre Processamento de Linguagem Natural.
    \item Estudo teórico sobre aprendizado de máquina aplicado à análise de textos.
\end{itemize}

\section{Metodologia}
\label{sub:metodologia}

Devido à natureza deste trabalho, nota-se que o modelo de pesquisa adequado deve possuir características tanto da pesquisa exploratória quanto da pesquisa experimental. Outro método de pesquisa que será utilizado é a pesquisa-ação, onde existirão ciclos de coleta e análise de dados. A cada ciclo, a análise dos dados do ciclo anterior servirão de insumo para tomadas de decisão no desenvolvimento do projeto.

Durante a fase de desenvolvimento do sistema, pretende-se utilizar uma abordagem de ágil, com a aplicação de algumas práticas adaptadas do \textit{framework} Scrum:

\begin{itemize}
    \item Nas retrospectivas de \textit{sprint}, serão realizadas avaliações, para identificar os pontos fortes e fracos da \textit{sprint} que terminou, bem como propor possíveis ações que poderão ser tomadas para mitigar os pontos fracos, na próxima \textit{sprint}.
    \item \textit{Daily meetings}, onde serão feitos comentários sobre o que foi realizado no dia e dúvidas poderão ser reportadas. Acontecerão diária e remotamente, para o melhor acompanhamento do orientador.
    \item Será definido um \textit{backlog} de produto e um para cada \textit{sprint}.
\end{itemize}

\section{Organização do Trabalho}
\label{sec:organização_do_trabalho}

Este trabalho está divido em quatro capítulos. No capítulo 2 são apresentadas as revisões bibliográficas realizadas para pré-processamento de textos e aprendizado de máquina aplicado a processamento de linguagem natural. No capítulo 3 são abordados os trabalhos relacionados, ferramentas utilizadas e o planejamento das atividades. O capítulo 3 apresenta os resultados alcançados com essa pesquisa. Por fim, o capítulo 5 possui as considerações finais, perspectivas futuras e um cronograma para a continuação deste trabalho.
