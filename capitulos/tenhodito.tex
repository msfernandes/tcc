\clearpage

\subsection{Tenho Dito}

``Tenho Dito'' é uma aplicação \textit{web}, desenvolvida utilizando a linguagem \textit{python} com o \textit{framework Django}, e tem como objetivo ser uma forma mais lúdica de visualização dos dados disponíveis nos \textit{webservices} de dados abertos da Câmara dos Deputados. Utiliza métodos de processamento de linguagem natural e aprendizado de máquina para extrair o perfil temático dos parlamentares, analisando o texto de seus discursos e proposições. Além disso, também é possível traçar os temas mais discutidos (tanto em propostas quanto nos próprios discursos) pelos deputados de uma determinada região ou por partidos.

A aplicação é divida em dois grandes módulos: \textit{nlp} e \textit{core}. O primeiro é responsável por todas as operações relacionadas ao processamento dos textos, o que inclui aprendizado de máquina. Já o segundo módulo é responsável pela parte \textit{web}. No momento de escrita desse trabalho, ainda não tinha sido implementado o segundo módulo. Entretanto, estão disponíveis alguns protótipos, que mostram as possíveis funcionalidades do sistema.

Conforme mencionado no item ``\textit{frameworks}'', em \ref{ferramentas}, a análise dos textos será realizada com o apoio das ferramentas:

\begin{itemize}
    \item \textbf{Plagiarism:} biblioteca desenvolvida pelo professor orientador do autor desse trabalho, Fábio Macêdo Mendes, e possui uma série de funcionalidades utilizadas no pré-processamento dos textos, como extração de \textit{tokens}, \textit{stemização}, remoção de \textit{stop words}, geração de \textit{n-gramas} e geração de \textit{bag-of-words} (com os diferentes tipos de representação dos termos, descrito na seção \ref{sec:representação_dos_termos} desse trabalho).
    \item \textbf{Textblob:} biblioteca \textit{python} para processamento de dados textuais. Ela fornece uma interface simples para realizar tarefas comuns do processamento de linguagem natural, como análise de sentimento e classificação, por exemplo. Utiliza a biblioteca \textit{LNTK} para realizar essas tarefas.
\end{itemize}

\subsubsection{Classificação dos Discursos}

A classificação dos discursos é dividida em duas etapas. A primeira etapa consiste em, inicialmente, dividir o texto em parágrafos, para que a análise seja realizada com uma quantidade menor de texto, e em seguida os parágrafos são classificados entre ``protocolo parlamentar'' ou ``conteúdo''. Por exemplo, o trecho ``É preciso haver quórum de 257 Srs. Deputados para aprovação da matéria, quórum mínimo. A votação é normal. Então, acho que, quando houver uns 300 ou 320 votos, encerraremos.'' não representa um conteúdo significativo, da mesma forma que ``O SR. ALCEU MOREIRA - Sr. Presidente, primeiro a medida provisória, logicamente.'' também não agregaria nenhum valor à análise. Trechos como esses devem se classificados como ``protocolo parlamentar'' e descartados da análise temática. A segunda etapa do processamento é a classificação temática dos parágrafos classificados como ``conteúdo'', na etapa anterior.

Para ambas etapas o procedimento adotado é o mesmo, com algumas alterações nos classificadores. Primeiro, um classificador \textit{NaiveBayesClassifier}, implementado pela biblioteca \textit{textblob}, é instanciado, utilizando dois conjuntos de palavras iniciais, um para definir ``protocolo parlamentar'' e outro para ``conteúdo'', como mostrado a seguir:

\begin{itemize}
    \item \textbf{Protocolo Parlamentar:} ``agradecimento agradeço muito obrigado v.exa. digníssimo nobre deputado amigo peço registro pela ordem pedir um aparte mérito emendas votado sessão comissão protocolo regimento pronunciamento divulgação''
    \item \textbf{Conteúdo:} ``educação universidade estudante professor ensino escola educador saúde médicos hospitais sus remédios atendimento hospitalar tratamento leitos religião templo igreja deus bíblia fé jesus segurança polícia crime violência punição arma contrabando ditadura militar golpe 31 de março tortura censura mulher aborto feminicídio feminismo feminista maria da penha petrobras pré-sal refinamento gasolina álcool combustível petrolão corrupção ministério público agu lava-jato mensalão impeachment crime de responsabilidade agronegócio agricultura agrícolas soja lavoura rural indústria desendustrialização empregos competitividade direitos humanos minorias tortura tráfego de pessoas trabalho escravo''
\end{itemize}

Em seguida, todos os parágrafos são classificados e, dentre os que foram classificados com uma probabilidade maior que 80\%, os 100 melhores colocados são utilizados para realizar o treinamento inicial do classificador. A partir disso, é realizado um treinamento supervisionado, onde o classificador sugere uma classe e um especialista diz se o trecho corresponde à classe sugerida, caso não seja ele deve fornecer a classe correta. Ao finalizar o treinamento supervisionado, todos os parágrafos são classificados novamente, agora com o classificador melhor treinado.

Com o resultado a primeira classificação, obtém-se um conjunto de parágrafos classificados como ``conteúdo'', que serão usados na classificação temática. De forma semelhante à primeira classificação, um classificador \textit{NaiveBayesClassifier} é instanciado, agora com um conjunto de palavras para cada tema:

\begin{itemize}
    \item \textbf{Agropecuária:} ``agropecuária fertilizantes agronegócio abate suínos ovos cabeças bovinos frangos exportação carne animal milho ração aviária laranja safra frutos pomares laranjeiras fazenda pés produzir hectares quilos fruta  produtor orgânico consumidor toneladas  embrapa bezerros pecuária veterinária filhotes sementes agro produção água sol área degradação produtor café importação agrícola pescador alimento alimentação açúcar ibge fertilizante lavouras grão bovino soja etanol frutos rural''
    \item \textbf{Saúde:} ``saúde médico doença vírus zika pesquisa paciente estudo mosquito epidemia chikungunya tratamento procedimento tremor causa gêmeos dengue transmissão cubano bebês cirurgia cientista risco sintomas dor ultrassom dr aegypt ovário microcefalia gravidez sistema imune imunológico drogas fertilização febre diagnóstico renal sangue insuficiente insuficiência cérebro idade nascimento hipotálamo morte dna corpo cardio muscular vacina''
    \item \textbf{Esporte:} ``esporte jogo jogador clube time contrato treino mundial atleta surf futebol disputa penalidade compo estádio ataque atacante bola goleiro treinador seleção técnico campeonato gol pontuação futsal vitória perde perdedor lutador torcedor torcida rival diretor falta conquista prorrogação empate surfista assistência ufc''
    \item \textbf{Educação:} ``educação estudo ensino escola médio prova enem universidade faculdade matemática avaliação aluno curso pesquisa inep exame pública mec professor redação criança texto reforma currículo curricular campus leitura literatura desempenho formação qualidade disciplina fies superior analfabeto analfabetismo português física química geometria''
    \item \textbf{Ciência e Tecnologia:} ``ciência tecnologia novidades empresa startup smart serviço smartphone consumidor produto google aparelho samsung celular internet inteligência artificial desenvolvimento dispositivo lançamento  aplicativo inovar inovação sony conectar conectado comunicação 3g 4g 5g iphone sistema telecomunicações satélite design científico artigo computador tráfego eletrônico apple whatsapp televisão tv telefone avanço espacial''
    \item \textbf{Economia:} ``economia trabalho crédito compra banco bilhões milhões vendas contas inflação consumidor juros queda crise taxa resultado econômico gasto pagamento valor financeiro investimento dinheiro índice comércio empresa desemprego fgts limite emprego cartão varejo déficite fundo recessão recuo salário lojista tesouro fiscal inadimplente recurso dólar euro moeda bolsa endividado projeções crescimento capital ações negócios''
    \item \textbf{Política:} ``política deputado congresso pt partido estado união reforma lei legislatura legislação pec pmdb aprovar voto bancada população senado senador câmara deputado sindicato candidato candidatura mandato comissão ministério constituição eleição eleições delação judiciário votações prefeitura prefeito vereador assembleia procurador corrupção''
    \item \textbf{Meio Ambiente:} ``ambiente área água rio empresa desastres multa seca barragem furacão desmatamento floresta tropical ibama parque preservação região terra planeta poluição ambiental espécie animais plantas platações petróleo emissão gás chuva temporal sol clima temperatura estufa aquecimento global umidade terremoto planeta biodiversidade biologia mar oceano calor energia sustentável madeira reflorestamento tempestade niño florescimento hídrico climática''
    \item \textbf{Direitos Humanos:} ``direitos humanos mulher tortura violência morte justiça onu sexual vítima sexual adolescente presídio prevenção união negro branco segurança refugiado homens humanitario conflito sociedade racismo sexismo machismo machista feminismo feminista defensoria estupro jovens criança prostituição assassinato liberdade idoso inclusão social preconceito gay homossexual heterosexual lgbt lésbica bissexual travesti transexual transgênero impunidade imigrante''
    \item \textbf{Segurança:} ``segurança ataque polícia suspeito morte crime terror rebelde investigação civil federal guerra onu vítima invasão preso presídio assassinato bombardeio apreensão incidente defesa exército marinha aeronáutica prisão ameaça bomba testemunha promotor policial tragédia assalto protesto''
\end{itemize}

Todos os parágrafos são classificados novamente e é gerado um conjunto com os melhores classificados, que é usado para realizar o treinamento inicial do classificador. E então acontece o treinamento supervisionado, onde um especialista diz se a classificação sugerida faz sentido e indica a classe correta quando não faz.

Também é possível realizar um treinamento não supervisionado para ambos os classificadores, de forma que as sugestões de classificação são utilizadas para o treinamento sem a análise de um especialista.

A cada iteração da fase de treinamento todas as probabilidades dos textos adicionados ao classificador são recalculadas, o que implica no aumento significativo do tempo de processamento.
