\clearpage
\section{Aprendizado Semi-Supervisionado}

Essa forma de aprendizado é útil quando quando existem apenas alguns exemplos já classificados e pode ser utilizado tanto em tarefas de classificação quanto em tarefas de \textit{clustering}. A ideia do aprendizado semi-supervisionado é utilizar esses exemplos previamente classificados para se obter informações sobre o problema e utilizá-las para auxiliar o processo de aprendizado a partir de exemplos não classificados \cite{bruce}.

Existem várias estratégias para definir algoritmos de aprendizado semi-supervisionado. É comum utilizar modelos de aprendizado não-supervisionado com alterações, afim de se poder inserir informações a priori.

Podemos tornar o algoritmo \textit{k-means} semi-supervisionado se fixarmos \textit{clusters} específicos para determinados termos previamente classificados, de forma que esse eles sempre pertençam aos \textit{clusters} pré-fixados. Dessa forma, a informação inserida influenciará diretamente na posição dos \textit{clusters}.

No \textit{LDA}, também podemos fixar algumas linhas da tabela de mistura \(Q\). Dessa forma, podemos dizer que um texto pertence a um tópico específico ou a misturas específicas de tópicos.
