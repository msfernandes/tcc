\subsection{Validação Cruzada}

A validação cruzada é uma técnica geralmente utilizada para avaliar modelos preditivos, buscando estimar o quão preciso é o modelo avaliado. A ideia principal da validação cruzada consiste em dividir o conjunto de dados em subconjuntos mutualmente exclusivos, onde alguns desses subconjuntos serão utilizados para a estimação dos parâmetros do modelo (treinamento) e os outros subconjuntos serão utilizados para a validação do modelo (validação ou teste) \cite{kohavi1995}.

O método \textit{holdout} é uma forma de validação de cruzada comumente utilizada e consiste em dividir o conjunto de dados em dois subconjuntos mutualmente exclusivos, onde seus tamanhos podem, ou não, ser diferentes. Um subconjunto servirá para o treinamento do modelo e o outro para a sua validação. A proporção mais comum é considerar 2/3 dos dados para treinamento e 1/3 para a validação. Essa abordagem é indicada quando está disponível uma grande quantidade de dados, quando o conjunto de dados é muito pequeno, o erro calculado na predição pode sofrer muita variação \cite{kohavi1995}.

Uma das métricas mais simples para avaliar o modelo é a comparação item a item dos resultados esperados pelos resultados obtidos:
%
\begin{align}
P_{a}=\frac{N_{vp}}{N_{it}},
\end{align}
%
onde:

\begin{itemize}
    \item \(P_{a}\) é a porcentagem de acertos do modelo,
    \item \(N_{vp}\) é o número de itens classificados corretamente pelo modelo e
    \item \(N_{it}\) é o número total de itens que fazem parte do conjunto de testes.
\end{itemize}
