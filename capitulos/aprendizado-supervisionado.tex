\section{Aprendizado Supervisionado}

No aprendizado supervisionado, cada exemplo de treinamento é descrito por um conjunto de atributos que servem como dados de entrada e são associados a um valor de saída. A partir de um conjunto pré-definido de entradas e saídas, o algoritmo consegue gerar uma saída adequada para uma nova entrada. O aprendizado supervisionado é a principal técnica utilizada para problemas de classificação e regressão \cite{mohri2012}.

\subsection{Aprendizado Bayesiano}

O aprendizado Bayesiano é um conjunto de técnicas baseadas em análise estatística que utilizam a fórmula de Bayes. Normalmente são métodos supervisionados, ainda que alguns algoritmos não-supervisionados possam ser mapeados em métodos Bayesianos \cite{mitchell1997}.

As principais vantagens do aprendizado Bayesiano são o fato de se poder embutir nas probabilidades calculadas o conhecimento de domínio (caso se tenha) e a capacidade das classificações feitas pelo algoritmo se basearem em evidências fornecidas e numa análise estatística bem fundamentada. Por outro lado, frequentemente envolvem o cálculo de médias e outras medidas estatísticas que pode ocasionar em um alto custo computacional.

\subsubsection{Classificador \textit{naive Bayes}}

Uma forma de mitigar a dificuldade de cálculo está em considerar modelos probabilísticos simplificados que permitem um tratamento analítico para as probabilidades calculadas \cite{pardo2002}.

O classificador Bayesiano ingênuo (ou \textit{naive Bayes}, em inglês), admite que os atributos do elemento a ser classificado são independentes entre si, dada a categoria da classificação \cite{pellucci2011}. Segundo \citeonline{oguri2007}, existem dois tipos principais de classificadores Bayesianos ingênuos utilizados em processamento de linguagem natural: o modelo binário e o modelo multinomial. Cada modelo está relacionado a um tipo de BOW. Também é possível tratar a BOW TF-IDF como um modelo Gaussiano adequado para variáveis contínuas \cite{hand2001}.

\paragraph{Modelo Binário}
O modelo binário representa um documento como um vetor binário, ou seja, o valor 0 em uma posição \(k\) (onde \(k\) representa uma palavra do documento) representa a não ocorrência do termo e o valor 1 representa ao menos uma ocorrência desse termo. Este modelo simplesmente especifica a probabilidade de ocorrência de cada termo.

\paragraph{Modelo Multinomial}
Já o modelo multinomial assume que o documento é representado por um vetor de inteiros, representando a quantidade de vezes que um termo ocorre no documento. Este modelo também especifica a probabilidade de ocorrência de um termo, mas permite ocorrências múltiplas.

Os classificadores Bayesianos são baseados na aplicação do Teorema de Bayes:
%
\begin{align}
P(classe|A) = \frac{P(classe) \times P(A|classe)}{P(A)},
\end{align}
%
onde:

\begin{itemize}
    \item \(P(classe)\) é a probabilidade da classe em questão, no contexto do teorema de Bayes, ela é comumente denominada probabilidade \textit{a priori} (ou \textit{prior})
    \item \(P(A|classe)\) é a probabilidade de obter um conjunto de dados \(A\) condicional a \(classe\). Isto é conhecido como \textit{likelihood} ou verossimilhança. O modelo Bayes ingênuo assume que \(P(A|classe) = \prod_{i} P(a_{i}|classe)\)
    \item \(P(classe|A)\) é a probabilidade de um elemento pertencer a uma classe dado um conjunto de observações \(A\). Conhecido como o \textit{a posteriori} na literatura Bayesiana
    \item \(P(A)\) é a probabilidade da nova instância a ser classificada. Este termo corresponde ao fator de normalização do posterior e é frequentemente ignorado. Na literatura Bayesiana é conhecido como evidência \cite{jaynes2003}
\end{itemize}

Para calcular a classe mais provável da nova instância, calcula-se as probabilidades de todas as classes possíveis e escolhe-se a classe com maior probabilidade. Em termos estatísticos, isso é equivalente a maximizar \(P(classe|A)\). Como muitas vezes utiliza-se um prior uniforme ao longo das classes, o problema matemático principal frequentemente consiste em encontrar o valor de máxima verossimilhança.

Considerando que \(A= ( a_1, a_2, \ldots , a_n)\), onde \(a_n\) são os atributos que compõe \(A\), a suposição ``ingênua'' que o classificador faz é que todos os atributos de \(A\) são independentes entre si, o que simplifica o cálculo da probabilidade de \(P(A|classe)\), podendo ser reduzida a \(P(a_1|classe) \times P(a_2|classe) \times\ \ldots \times P(a_n|classe)\). Logo,
%
\begin{align}
P(classe|A) \propto P(classe) \times \prod_{i=1}^{n} P(a_i|classe)
\end{align}

Em problemas de NLP, \(a_i\) usualmente correspondem aos valores contidos em uma BOW, ainda que seja possível adicionar outros tipos de atributos para a análise.
