\chapter*[Introdução]{Introdução}
\addcontentsline{toc}{chapter}{Introdução}


\section{Contextualização}
\label{sec:contextualização}

O desenvolvimento de novas ferramentas de interação entre governo e sociedade é fundamental para o avanço da democracia \cite{consegi2011}. Porém, torna-se imprecindível a aplicação do conceito de Dados Abertos por parte do Governo para que essas novas ferramentas tenham efeitos significativos para a sociedade.

O conceito para Dado Aberto a ser considerado para esse trabalho é o definido pela \textit{Open Knowledge}, onde é estabelecido que um dado (ou um conhecimento) é aberto quando estiver livre para uso, reuso e redistribuição \cite{open_knowledge}. Ou seja, a informação deve estar disponível à todos, sem restrições de \textit{copyright}, patentes ou outros mecanismos de controle.

No contexto governamental, os Dados Abertos fortalecem três características indispensáveis para a democracia: transparência, participação e colaboração \cite{consegi2011}. Onde a transparência tem o papel de informar a sociedade sobre as ações que estão sendo tomadas ou que serão tomadas pelo governo, a participação permite que os cidadãos auxiliem o poder público a elaborar políticas mais eficazes e a colaboração entre a sociedade, diferentes níveis de governo e a iniciativa privada aprimoram a eficácia do Estado.

De acordo com a Lei n\º 12.527/2011, também conhecida como Lei de Acesso à Informação, qualquer cidadão, sem necessidade de justificativa, pode solicitar dados ou informações à qualquer órgão ou entidade pública dos poderes Executivo, Legislativo e Judiciário, além do Ministério Público, nas esferas Federal, Estadual e Municipal \cite{lei_acesso_informacao}. Para atender à Lei mencionada anteriormente, a Câmara dos Deputados criou um portal que tem como objetivo disponibilizar dados brutos para a utilização em aplicações que permitam a percepção mais efetiva das atividades parlamentares, desenvolvidas pelos cidadãos e entidades da sociedade civil \cite{camara_dados_abertos}.

\section{Objetivos}
\label{sec:objetivos}

\subsection{Objetivo Geral}
\label{sub:objetivo_geral}

O objetivo desse trabalho é utilizar técnicas de processamento de linguagem natural para, através da análise dos discursos e proposições dos parlamentares, determinar um perfil temático para os Deputados, bem como evidenciar o termos mais utilizados em seus discursos e proposições e, assim, permitir que o cidadão veja, de forma comparativa e por meio de um \textit{website}, o que seus representantes no parlamento mais dizem em seus discursos e o que mais dizem em suas proposições.

\subsection{Objetivos Específicos}
\label{sub:objetivos_específicos}

\begin{itemize}
    \item Avaliar o conjunto de dados existentes no portal de Dados Abertos da Câmara dos Deputados, afim de definir quais informações serão utilizadas para o desenvolvimento da solução.
    \item Desenvolver uma biblioteca Python para consumir os dados do \textit{webservice} da Cãmara dos Deputados.
    \item Desenvolver uma biblioteca Python/Django para persistir em banco de dados as informações provenientes do \textit{webservice} da Câmara dos Deputados.
    \item Estabelecer uma metodologia de desenvolvimento, baseada em boas práticas da Engenharia de Software, para conduzir esse trabalho no que diz respeito ao processo investigativo, implementação e avaliação da solução.
    \item Orientar-se por estudos na área de Processamento de Linguagem Natural e Mineração de Dados, visando a aplicação de algorítmos pertinentes à solução. Dentre esses algorítmos, pretende-se investigar a utilização de \textit{bag of words} e o método de clusterização \textit{k-means}.
\end{itemize}