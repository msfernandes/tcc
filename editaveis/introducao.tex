\chapter*[Introdução]{Introdução}
\addcontentsline{toc}{chapter}{Introdução}

O desenvolvimento de novas ferramentas de interação entre governo e sociedade é fundamental para o avanço da democracia \cite{consegi2011}. Porém, torna-se imprecindível a aplicação do conceito de Dados Abertos por parte do Governo para que essas novas ferramentas tenham efeitos significativos para a sociedade.

O conceito para Dado Aberto a ser considerado para esse trabalho é o definido pela \textit{Open Knowledge}, onde é estabelecido que um dado (ou um conhecimento) é aberto quando estiver livre para uso, reuso e redistribuição \cite{open_knowledge}. Ou seja, a informação deve estar disponível à todos, sem restrições de \textit{copyright}, patentes ou outros mecanismos de controle.

No contexto governamental, os Dados Abertos fortalecem três características indispensáveis para a democracia: transparência, participação e colaboração \cite{consegi2011}. Onde a transpParência tem o papel de informar a sociedade sobre as ações que estão sendo tomadas ou que serão tomadas pelo governo, a participação permite que os cidadãos auxiliem o poder público a elaborar políticas mais eficazes e a colaboração entre a sociedade, diferentes níveis de governo e a iniciativa privada aprimoram a eficácia do Estado.

O projeto Dados Abertos da Câmara dos Deputados tem como objetivo disponibilizar dados brutos para a utilização em aplicações que permitam a percepção mais efetiva das atividades parlamentares, desenvolvidas pelos cidadãos e entidades da sociedade civil \cite{camara_dados_abertos}. Dentre os dados disponíveis no \textit{webservice} da Câmara dos Deputados, serão utilizados, nesse trabalho, informações sobre os Deputados, Partidos, Bancadas, Proposições e Discursos em plenário.