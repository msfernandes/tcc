\begin{resumo}
O processamento de linguagem natural tem sido utilizado com sucesso na área de análise de discurso, onde é possível reconhecer padrões e classificar textos, extraindo informações de grandes volumes de dados. Este trabalho tem como objetivo extrair o perfil temático dos deputados federais, através do processamento dos textos obtidos de seus discursos e proposições, bem como desenvolver uma aplicação \textit{web} para que os resultados dessa pesquisa sejam apresentados de forma lúdica e amigável. O texto discute as técnicas de processamento de linguagem natural utilizadas nesta análise, que incluem a remoção de \textit{stop words}, algumas técnicas de \textit{stemização}, representação dos textos em \textit{bag-of-words} e algumas técnicas de aprendizagem de máquina supervisionada e não-supervisionada, como \textit{Naive Bayes} e \(k\)-means.

 \vspace{\onelineskip}

 \noindent
 \textbf{Palavras-chaves}: Processamento de linguagem natural, aprendizado de máquina.
\end{resumo}
