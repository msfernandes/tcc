\begin{resumo}[Abstract]
 \begin{otherlanguage*}{english}
   Natural language processing has been used successfully in the discourse analysis where is possible to recognize patterns and classify texts, extracting information from large volume of data. This paper aims to extract the thematic profile of the federal deputies through the processing of texts obteined from their speeches and proposals, as well as develop a web application so that the results of this research are presented in a playful and friendly way. The text discusses the natural language processing techniques used in this analysis, which include the removal of stop words, some stemming techniques, representation of texts in bag-of-words model and some techniques of supervised and unsupervised machine learning, such as Naive Bayes and k-means.

   \vspace{\onelineskip}

   \noindent
   \textbf{Key-words}: Natural language processing, machine learning.
 \end{otherlanguage*}
\end{resumo}
